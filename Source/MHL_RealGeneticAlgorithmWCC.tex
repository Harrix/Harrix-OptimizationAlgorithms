\subsection{Генетический алгоритм для решения задач на вещественных строках, в котором есть только два вида скрещивания: одноточечное и двухточечное скрещивание с возможностью полного копирования одного из родителей}\label{HarrixOptimizationAlgorithms:MHL_RealGeneticAlgorithmWCC}

\textbf{Тип алгоритма}: исследовательский алгоритм оптимизации.

\textbf{Идентификатор:} MHL\_RealGeneticAlgorithmWCC.

\textbf{Название:} генетический алгоритм для решения задач на вещественных строках, в котором есть только два вида скрещивания: одноточечное и двухточечное скрещивание с возможностью полного копирования одного из родителей.

Основан на стандартном генетическом алгоритме на вещественных строках:  \href{https://github.com/Harrix/Standard-Genetic-Algorithm}{https://github.com/Harrix/Standard-Genetic-Algorithm}. 

Отличается от стандартного генетического алгоритма тем, что есть только два вида скрещивания: одноточечное и двухточечное скрещивание с возможностью полного копирования одного из родителей. Равномерное скрещивание отсутствует. То есть данным алгоритмом проверяем: есть ли разница в эффективности алгоритма, если точки разрыва при скрещивании делать и по краям родителей, а не только внутри хромосомы.

В качестве операторов-заменителей используются:
\begin{itemize}
\item SinglepointCrossoverWithCopying --- одноточечное скрещивание с возможностью полного копирования одного из родителей;
\item TwopointCrossoverWithCopying --- двухточечное скрещивание с возможностью полного копирования одного из родителей;
\end{itemize}

Подробно прочитать о этих двух операторах с формулами и примерами можно тут:

\href{https://github.com/Harrix/HarrixSetOfOperatorsAlgorithms}{https://github.com/Harrix/HarrixSetOfOperatorsAlgorithms}

Число вариантов настроек алгоритма равно \textbf{72}.

Результат исследований алгоритма можно посмотреть тут:

\href{https://github.com/Harrix/HarrixDataOfOptimizationTesting}{https://github.com/Harrix/HarrixDataOfOptimizationTesting}

В библиотеке HarrixMathLibrary данный алгоритм реализован в вде функции MHL\_RealGeneticAlgorithmWCC. Библиотеку можно найти тут:

\href{https://github.com/Harrix/HarrixMathLibrary}{https://github.com/Harrix/HarrixMathLibrary}