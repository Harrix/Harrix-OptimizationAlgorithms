\subsection{Генетический алгоритм для решения задач на бинарных строках с изменяющимся соотношением числа поколений и размера популяции}\label{HarrixOptimizationAlgorithms:HML_BinaryGeneticAlgorithmWDPOfNOfGPS}

\textbf{Тип алгоритма}: исследовательский алгоритм оптимизации.

\textbf{Идентификатор:} HML\_BinaryGeneticAlgorithmWDPOfNOfGPS.

\textbf{Название:} генетический алгоритм для решения задач на бинарных строках с изменяющимся соотношением числа поколений и размера популяции.

Основан на стандартном генетическом алгоритме на бинарных строках:  \href{https://github.com/Harrix/Standard-Genetic-Algorithm}{https://github.com/Harrix/Standard-Genetic-Algorithm}. 

Отличается от стандартного генетического алгоритма, тем, что размер популяции и число поколений рассчитывается из числа вычислений целевой функции не как одинаковые величины (извлечение квадратного корня), а через некоторое соотношение.

Число поколений определяется по формуле:
\begin{equation}
NumberOfGenerations = int \left( CountOfFitness^{Proportion}\right).
\end{equation}

Число поколений, соответственно, определяется по формуле:
\begin{equation}
PopulationSize = int \left( \dfrac{CountOfFitness}{NumberOfGenerations}\right).
\end{equation}


Тут $CountOfFitness$ --- максимальное число вычислений целевой функции, а $Proportion$ --- \textbf{новый} параметр в алгоритме, который обозначает <<долю>> числа поколений от общего числа вычислений целевой функции.

$Proportion$ может принимать значения в интервале $ [0;1] $, а именно:
\begin{equation}
Proportion \in \left\lbrace 0; 0.1; 0.2; 0.3; 0.4; 0.5; 0.6; 0.7; 0.8; 0.9; 1\right\rbrace .
\end{equation}

То есть $Proportion$ может принимать $ 11 $ значений.

По сравнению с стандартным генетическим алгоритмом число вариантов настроек алгоритма увеличивается в 11 раз и равно \textbf{594}.

Чем меньше Proportion, тем меньше будет число поколений.

При $ Proportion=0.5 $ получим обычный стандартный генетический алгоритм. Число поколений будет равно $ \sqrt{CountOfFitness}$ (без учета получения целой части числа).

При $ Proportion=0 $ число поколений будет равно $ 1 $.

При $ Proportion=1 $ число поколений будет равно $ CountOfFitness $.

Результат исследований алгоритма можно посмотреть тут:

\href{https://github.com/Harrix/HarrixDataOfOptimizationTesting}{https://github.com/Harrix/HarrixDataOfOptimizationTesting}

В библиотеке HarrixMathLibrary данный алгоритм реализован в вде функции HML\_BinaryGeneticAlgorithmWDPOfNOfGPS. Библиотеку можно найти тут:

\href{https://github.com/Harrix/HarrixMathLibrary}{https://github.com/Harrix/HarrixMathLibrary}