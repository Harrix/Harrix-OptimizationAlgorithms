\subsection{Генетический алгоритм с двойным количеством поколений для решения задач на вещественных строках, где на четных поколениях целевая функция высчитывается как среднеарифметическое родителей}\label{HarrixOptimizationAlgorithms:MHL_RealGeneticAlgorithmTwiceGenerations}

\textbf{Тип алгоритма}: исследовательский алгоритм оптимизации.

\textbf{Идентификатор:} MHL\_RealGeneticAlgorithmTwiceGenerations.

\textbf{Название:} генетический алгоритм с двойным количеством поколений для решения задач на вещественных строках, где на четных поколениях целевая функция высчитывается как среднеарифметическое родителей.

Основан на стандартном генетическом алгоритме на вещественных строках:  \href{https://github.com/Harrix/Standard-Genetic-Algorithm}{https://github.com/Harrix/Standard-Genetic-Algorithm}. 

Отличается от стандартного генетического алгоритма, тем, что количество популяций в два раза больше, чем в стандартном, при сохранении остальных параметров. Чтобы не превышать число вычислений целевой функции, на четных поколениях (считается, что первое --- это инициализация ГА) значения целевой функции вычисляются как среднеарифметические родителей. При этом в учете лучшего решения учитываются только решения из нечетных поколений, где целевая функция вычисляется правильно. На самом деле число поколений не строго в два раза больше, чем в сГА, а в два раза больше минус один. Это объясняется, что последнее поколение будет поколением, где считаются значения целевых функций как среднеарифметические родителей, а, следовательно, лишнее. Хотя можно это лишнее поколение и прокрутить, но оно не будет влиять на выдаваемое решение.

Число вариантов настроек алгоритма равно \textbf{108}.

Результат исследований алгоритма можно посмотреть тут:

\href{https://github.com/Harrix/HarrixDataOfOptimizationTesting}{https://github.com/Harrix/HarrixDataOfOptimizationTesting}

В библиотеке HarrixMathLibrary данный алгоритм реализован в вде функции MHL\_RealGeneticAlgorithmTwiceGenerations. Библиотеку можно найти тут:

\href{https://github.com/Harrix/HarrixMathLibrary}{https://github.com/Harrix/HarrixMathLibrary}