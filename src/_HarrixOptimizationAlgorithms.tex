\documentclass[a4paper,12pt]{article}

\input{packages}
\input{styles}

\title{HarrixOptimizationAlgorithms. Сборник описаний алгоритмов оптимизации нулевого порядка. v. 1.7}
\author{А.\,Б. Сергиенко}
\date{\today}


\begin{document}

\input{names}

\maketitle

\begin{abstract}
В данном документе дано собрано множество описаний стандартных, нестандартных алгоритмов, модификаций стандартных. Здесь приведено лишь описание алгоритмов, а не их исследование эффективности. Большинство алгоритмов неэффективны. Это своеобразная <<свалка>>  алгоритмов, которые используются автором. На данный документ можно ссылаться в своих работах, чтобы указать, что тот или иной алгоритм оптимизации подробно описан в этом документе. Тут нет исследований эффективности алгоритмов с данными операторами --- это задача иных проектов. Здесь представлено только описание алгоритмов.
\end{abstract}

\tableofcontents

\newpage

\section{Введение}

Это своеобразная <<свалка>> алгоритмов оптимизации, которые используются автором. Большинство алгоритмов неэффективны. Здесь они приведены, чтобы можно было ссылаться на них.

Данный документ представляет его версию 1.6 от \today

Последнюю версию документа можно найти по адресу:

\href{https://github.com/Harrix/HarrixOptimizationAlgorithms}{https://github.com/Harrix/HarrixOptimizationAlgorithms}

С автором можно связаться по адресу \href{mailto:sergienkoanton@mail.ru}{sergienkoanton@mail.ru} или  \href{http://vk.com/harrix}{http://vk.com/harrix}.

Сайт автора, где публикуются последние новости: \href{http://blog.harrix.org/}{http://blog.harrix.org/}, а проекты располагаются по адресу \href{http://harrix.org/}{http://harrix.org/}.

\input{section_symbols}

\section{Некоторая вводная информация}\label{HarrixOptimizationAlgorithms:first_informamtion}

В каждом классе решаемых задач (задачи бинарной оптимизации, задачи вещественной оптимизации и др.) определен некий основной алгоритм. Обычно им является стандартный генетический алгоритм. И с ним сравниваются все остальные алгоритмы оптимизации, чтобы можно было выявить лучший алгоритм на множестве тестовых задач при определенных фиксированных настройках. Алгоритмы, которые ср

Алгоритмы представленные в данной работе бывают нескольких типов, которые описаны ниже. 

\textbf{Основной алгоритм оптимизации} --- некий алгоритм в классе решаемых задач (задачи бинарной оптимизации, задачи вещественной оптимизации и др.) относительно которого производится сравнение всех остальных алгоритмов. Данный алгоритм может со временем меняться. Это происходит в случае, если обнаруживается алгоритм, который по эффективности превосходит (или не хуже) предыдущий основной алгоритм оптимизации по всем параметрам на всех тестовых функциях.

\textbf{Сравниваемый алгоритм оптимизации} ---  некий алгоритм, который сравнивается по эффективности с основным алгоритмом и другими сравниваемыми алгоритмами по эффективности.

\textbf{Добавочный алгоритм оптимизации} --- алгоритм оптимизации, который не сравнивается по эффективности с основным алгоритмом и другими сравниваемыми алгоритмами по эффективности. Этот алгоритм является промежуточным, и в нем проверяется эффективность какой-нибудь настройки алгоритма. Например, в стандартном генетическом алгоритме есть три вида скрещивания: одноточечное, двухточечное и равномерное. А мы решили проверить трех точечное скрещивание. Для этого создает добавочный алгоритм, в котором есть только один вид скрещивания --- трехточечным, и проводим полное тестирование алгоритма. И в сравнении с обычным алгоритмом можем оценить  эффектность данного оператора. Если покажет эффектность, то уже можем создать сравниваемый алгоритм, который или уберет какой-то параметр или внесет трехточечное скрещивание на равноправных правах с другими видами скрещивания, или же, если на всех тестовых задачах трехточечное скрещивание покажет преимущество, то добавочный алгоритм станет сравниваемым алгоритмом. При этом отметим, что если просто добавим этот оператор в наравне с другими операторами, то нам не нужно будет пересчитывать весь алгоритм, так как просто добавим исследования из предыдущего исследования основного алгоритма.

\textbf{Исследовательский алгоритм оптимизации} --- также алгоритм оптимизации, который не сравнивается по эффективности с основным алгоритмом и другими сравниваемыми алгоритмами по эффективности. Его особенность, что в этом алгоритме <<вшито>> множество разных настроек, эффективность которых мы не знаем. Мы проводим полное исследование данного алгоритма, убираем неэффективные настройки или комбинации настроек и формируем уже сравниваемый алгоритм оптимизации.

\section{Постановка задачи оптимизации}\label{HarrixOptimizationAlgorithms:section_problemoptimization}

Рассмотрим постановку задачи оптимизации в общем случае, решаемую алгоритмами оптимизации.

Необходимо найти:
\begin{equation}
\label{HarrixOptimizationAlgorithms:eq:problemoptimization}
\bar{x}_{max} = \arg{ \max_{\bar{x} \in X}{f\left ( \bar{x} \right )} }\text {, где}
\end{equation}
\begin{equation*}
g_i\left (\bar{x}\right )\leq 0, i=\overline{1,m_1},
\end{equation*}
\begin{equation*}
h_j\left (\bar{x}\right )= 0, j=\overline{1,m_2}.
\end{equation*}

Здесь:

$ X $ --- множество всех возможных решений,

$ f\left ( \bar{x} \right ) $ --- функционал, определенный на данном множестве, возвращающий действительное число из интервала $ (-\infty;\infty) $,

$ m_1 $ --- число ограничений в виде неравенств,

$ m_2 $ --- число ограничений в виде равенств.

$ \bar{x} \in X $ имеет вид 
\begin{equation}
\label{StandardGA:eq:formofvector}
\bar{x}={\left(x_1;\dots;x_i;\dots;x_n \right)}^\mathrm{T}.
\end{equation}

В случае, если $ m_1=0 $ и $ m_2=0 $, имеем задачу безусловной оптимизации. В противном случае --- условной оптимизации.

В случае, если $ X $ --- множество всех бинарных векторов длины $ n $ таких что $ x_i\in\left\lbrace 0;1\right\rbrace  (i\hmm=\overline{1,n})$, то имеем задачу бинарной оптимизации (мощность поискового пространства $ X $ равна $ \mu(X)=2^n $). Если $ X $ --- множество всех вещественных векторов длины $ n $ таких что $ x_i\in\left\lbrace Left_i;Right_i\right\rbrace (i=\overline{1,n}) $, то имеем задачу вещественной оптимизации  и т.~д.

Будем предполагать в дальнейшем, что $ f(\bar{x}) $ может представлять собой  многоэкстремальный функционал, и вычислению подлежат только значения $ f(\bar{x}) $ (нет возможности вычислить производные от функционала и т.~д.).

Если требуется найти минимум функционала $ f\left ( \bar{x} \right ) $, то рещаем задачу нахождения максимума функционала $ -f\left ( \bar{x} \right ) $.

Также обычно при решении задачи оптимизации нам дано число $ CountOfFitness $ --- максимальное число вычислений целевой функции, которое нам дозволено использовать при запуске того или иного алгоритма оптимизации.


\section{Стандартный генетический алгоритм}\label{HarrixOptimizationAlgorithms:GA}

\subsection{Стандартный генетический алгоритм для решения задач на бинарных строках}\label{HarrixOptimizationAlgorithms:HML_StandartBinaryGeneticAlgorithm}

\textbf{Тип алгоритма}: основной алгоритм оптимизации.

\textbf{Идентификатор:} HML\_StandartBinaryGeneticAlgorithm.

\textbf{Название:} стандартный генетический алгоритм для решения задач на бинарных строках.

Подробное описание алгоритма представлено в данном проекте: 

\href{https://github.com/Harrix/Standard-Genetic-Algorithm}{https://github.com/Harrix/Standard-Genetic-Algorithm}.

Число вариантов настроек алгоритма равно \textbf{54}.

Результат исследований алгоритма можно посмотреть тут:

\href{https://github.com/Harrix/HarrixDataOfOptimizationTesting}{https://github.com/Harrix/HarrixDataOfOptimizationTesting}

В библиотеке HarrixMathLibrary данный алгоритм реализован в вде функции HML\_StandartBinaryGeneticAlgorithm. Библиотеку можно найти тут:

\href{https://github.com/Harrix/HarrixMathLibrary}{https://github.com/Harrix/HarrixMathLibrary}
\subsection{Стандартный генетический алгоритм для решения задач на вещественных строках}\label{HarrixOptimizationAlgorithms:HML_StandartRealGeneticAlgorithm}

\textbf{Тип алгоритма}: основной алгоритм оптимизации.

\textbf{Идентификатор:} HML\_StandartRealGeneticAlgorithm.

\textbf{Название:} стандартный генетический алгоритм для решения задач на вещественных строках.

Подробное описание алгоритма представлено в данном проекте:

\href{https://github.com/Harrix/Standard-Genetic-Algorithm}{https://github.com/Harrix/Standard-Genetic-Algorithm}. 

Число вариантов настроек алгоритма равно \textbf{108}.

Результат исследований алгоритма можно посмотреть тут:

\href{https://github.com/Harrix/HarrixDataOfOptimizationTesting}{https://github.com/Harrix/HarrixDataOfOptimizationTesting}

В библиотеке HarrixMathLibrary данный алгоритм реализован в вде функции HML\_StandartBinaryGeneticAlgorithm. Библиотеку можно найти тут:

\href{https://github.com/Harrix/HarrixMathLibrary}{https://github.com/Harrix/HarrixMathLibrary}

\section{Модификации генетического алгоритма}\label{HarrixOptimizationAlgorithms:ModGA}

\input{HML_BinaryGeneticAlgorithmWDPOfNOfGPS}
\subsection{Генетический алгоритм для решения задач на вещественных строках с изменяющимся соотношением числа поколений и размера популяции}\label{HarrixOptimizationAlgorithms:HML_RealGeneticAlgorithmWDPOfNOfGPS}

\textbf{Тип алгоритма}: исследовательский алгоритм оптимизации.

\textbf{Идентификатор:} HML\_RealGeneticAlgorithmWDPOfNOfGPS.

\textbf{Название:} генетический алгоритм для решения задач на вещественных строках с изменяющимся соотношением числа поколений и размера популяции.

Основан на стандартном генетическом алгоритме на вещественных строках:  \href{https://github.com/Harrix/Standard-Genetic-Algorithm}{https://github.com/Harrix/Standard-Genetic-Algorithm}. 

Отличается от стандартного генетического алгоритма, тем, что размер популяции и число поколений рассчитывается из числа вычислений целевой функции не как одинаковые величины (извлечение квадратного корня), а через некоторое соотношение.

Число поколений определяется по формуле:
\begin{equation}
NumberOfGenerations = int \left( CountOfFitness^{Proportion}\right).
\end{equation}

Число поколений, соответственно, определяется по формуле:
\begin{equation}
PopulationSize = int \left( \dfrac{CountOfFitness}{NumberOfGenerations}\right).
\end{equation}


Тут $CountOfFitness$ --- максимальное число вычислений целевой функции, а $Proportion$ --- \textbf{новый} параметр в алгоритме, который обозначает <<долю>> числа поколений от общего числа вычислений целевой функции.

$Proportion$ может принимать значения в интервале $ [0;1] $, а именно:
\begin{equation}
Proportion \in \left\lbrace 0; 0.1; 0.2; 0.3; 0.4; 0.5; 0.6; 0.7; 0.8; 0.9; 1\right\rbrace .
\end{equation}

То есть $Proportion$ может принимать $ 11 $ значений.

По сравнению с стандартным генетическим алгоритмом число вариантов настроек алгоритма увеличивается в 11 раз и равно \textbf{1188}.

Чем меньше Proportion, тем меньше будет число поколений.

При $ Proportion=0.5 $ получим обычный стандартный генетический алгоритм. Число поколений будет равно $ \sqrt{CountOfFitness}$ (без учета получения целой части числа).

При $ Proportion=0 $ число поколений будет равно $ 1 $.

При $ Proportion=1 $ число поколений будет равно $ CountOfFitness $.

Результат исследований алгоритма можно посмотреть тут:

\href{https://github.com/Harrix/HarrixDataOfOptimizationTesting}{https://github.com/Harrix/HarrixDataOfOptimizationTesting}

В библиотеке HarrixMathLibrary данный алгоритм реализован в вде функции HML\_RealGeneticAlgorithmWDPOfNOfGPS. Библиотеку можно найти тут:

\href{https://github.com/Harrix/HarrixMathLibrary}{https://github.com/Harrix/HarrixMathLibrary}
\input{HML_BinaryGeneticAlgorithmWDTS}
\input{HML_RealGeneticAlgorithmWDTS}
\input{HML_BinaryGeneticAlgorithmWCC}
\input{HML_RealGeneticAlgorithmWCC}
\input{HML_BinaryGeneticAlgorithmTournamentSelectionWithReturn}
\subsection{Генетический алгоритм для решения задач на вещественных строках с турнирной селекцией с возвращением, где размер турнира изменяется от 2 до размера популяции}\label{HarrixOptimizationAlgorithms:HML_RealGeneticAlgorithmTournamentSelectionWithReturn}

\textbf{Тип алгоритма}: добавочный алгоритм оптимизации.

\textbf{Идентификатор:} HML\_RealGeneticAlgorithmTournamentSelectionWithReturn.

\textbf{Название:} генетический алгоритм для решения задач на вещественных строках с турнирной селекцией с возвращением, где размер турнира изменяется от 2 до размера популяции.

Основан на стандартном генетическом алгоритме на вещественных строках:  \href{https://github.com/Harrix/Standard-Genetic-Algorithm}{https://github.com/Harrix/Standard-Genetic-Algorithm}. 

Отличается от стандартного генетического алгоритма, тем, что присутствует только турнирная селекция, но и она измененная, и размер турнира может изменяться.

Подробное описание турнирной селекции с возвращением можно прочитать тут:

\href{https://github.com/Harrix/HarrixSetOfOperatorsAlgorithms}{https://github.com/Harrix/HarrixSetOfOperatorsAlgorithms}


Так как основан на стандартном генетическом алгоритме, то размер популяции вычисляется, как квадратный корень из максимального числа вычислений целевой функции. Поэтому размер турнира $SizeOfTournament$ может теоретически  изменяться в пределах:

\begin{equation}
SizeOfTournament = \overline{1,int\left( \sqrt{CountOfFitness}\right) }.
\end{equation}

Тут $CountOfFitness$ --- максимальное число вычислений целевой функции.

$SizeOfTournament$ может принимать следующие значения в данном алгоритме:
\begin{equation}
SizeOfTournament \in \begin{Bmatrix}
2\\ 
3\\ 
4\\ 
5\\ 
1/3\text{ от популяции}\\ 
1/2\text{ от популяции}\\ 
2/3\text{ от популяции}\\ 
\text{Вся популяция} 
\end{Bmatrix}
\end{equation}
Если записывать строго, то получится следующее множество:
\begin{equation}
SizeOfTournament \in \begin{Bmatrix}
2\\ 
3\\ 
4\\ 
5\\ 
int\left( \frac{1}{3}\cdot  \sqrt{CountOfFitness}\right)  \\ 
int\left( \frac{1}{2}\cdot  \sqrt{CountOfFitness}\right)\\ 
int\left( \frac{2}{3}\cdot  \sqrt{CountOfFitness}\right)\\ 
int\left( \sqrt{CountOfFitness}\right)
\end{Bmatrix}
\end{equation}

То есть $SizeOfTournament$ может принимать $ 8 $ значений.

Число вариантов настроек алгоритма равно \textbf{288}.

Результат исследований алгоритма можно посмотреть тут:

\href{https://github.com/Harrix/HarrixDataOfOptimizationTesting}{https://github.com/Harrix/HarrixDataOfOptimizationTesting}

В библиотеке HarrixMathLibrary данный алгоритм реализован в виде функции HML\_RealGeneticAlgorithmTournamentSelectionWithReturn. Библиотеку можно найти тут:

\href{https://github.com/Harrix/HarrixMathLibrary}{https://github.com/Harrix/HarrixMathLibrary}
\input{HML_BinaryGeneticAlgorithmTwiceGenerations}
\input{HML_RealGeneticAlgorithmTwiceGenerations}

\section{Алгоритмы Монте-Карло}\label{HarrixOptimizationAlgorithms:MonteCarlo}
\subsection{Метод Монте-Карло (Monte-Carlo) для решения задач на бинарных строках}\label{HarrixOptimizationAlgorithms:HML_BinaryMonteCarloAlgorithm}

\textbf{Тип алгоритма}: сравниваемый алгоритм оптимизации.

\textbf{Идентификатор:} HML\_BinaryMonteCarloAlgorithm.

\textbf{Название:} Метод Монте-Карло (Monte-Carlo) для решения задач на бинарных строках.

Наиболее простой алгоритм. Случайно генерируется по равномерному закону распределения $ CountOfFitness $ решений из допустимой области. Из них выбирается лучшее.

При сравнении с работой, например, стандартного генетического алгоритма $ CountOfFitness $ (число вычислений целевой функции) выбирается таким же, что и у стандартного генетического алгоритма.

Результат исследований алгоритма можно посмотреть тут:

\href{https://github.com/Harrix/HarrixDataOfOptimizationTesting}{https://github.com/Harrix/HarrixDataOfOptimizationTesting}

В библиотеке HarrixMathLibrary данный алгоритм реализован в вде функции HML\_BinaryMonteCarloAlgorithm. Библиотеку можно найти тут:

\href{https://github.com/Harrix/HarrixMathLibrary}{https://github.com/Harrix/HarrixMathLibrary}
\input{HML_RealMonteCarloAlgorithm}


% Список литературы
\addcontentsline{toc}{section}{Список литературы}
\bibliographystyle{utf8gost71u}  %% стилевой файл для оформления по ГОСТу
\bibliography{biblio}     %% имя библиографической базы (bib-файла)
\newpage

\end{document}