\subsection{Генетический алгоритм для решения задач на бинарных строках с турнирной селекцией с возвращением, где размер турнира изменяется от 2 до размера популяции}\label{HarrixOptimizationAlgorithms:HML_BinaryGeneticAlgorithmTournamentSelectionWithReturn}

\textbf{Тип алгоритма}: добавочный алгоритм оптимизации.

\textbf{Идентификатор:} HML\_BinaryGeneticAlgorithmTournamentSelectionWithReturn.

\textbf{Название:} генетический алгоритм для решения задач на бинарных строках с турнирной селекцией с возвращением, где размер турнира изменяется от 2 до размера популяции.

Основан на стандартном генетическом алгоритме на бинарных строках:  \href{https://github.com/Harrix/Standard-Genetic-Algorithm}{https://github.com/Harrix/Standard-Genetic-Algorithm}. 

Отличается от стандартного генетического алгоритма, тем, что присутствует только турнирная селекция, но и она измененная, и размер турнира может изменяться.

Подробное описание турнирной селекции с возвращением можно прочитать тут:

\href{https://github.com/Harrix/HarrixSetOfOperatorsAlgorithms}{https://github.com/Harrix/HarrixSetOfOperatorsAlgorithms}


Так как основан на стандартном генетическом алгоритме, то размер популяции вычисляется, как квадратный корень из максимального числа вычислений целевой функции. Поэтому размер турнира $SizeOfTournament$ может теоретически  изменяться в пределах:

\begin{equation}
SizeOfTournament = \overline{1,int\left( \sqrt{CountOfFitness}\right) }.
\end{equation}

Тут $CountOfFitness$ --- максимальное число вычислений целевой функции.

$SizeOfTournament$ может принимать следующие значения в данном алгоритме:
\begin{equation}
SizeOfTournament \in \begin{Bmatrix}
2\\ 
3\\ 
4\\ 
5\\ 
1/3\text{ от популяции}\\ 
1/2\text{ от популяции}\\ 
2/3\text{ от популяции}\\ 
\text{Вся популяция} 
\end{Bmatrix}
\end{equation}
Если записывать строго, то получится следующее множество:
\begin{equation}
SizeOfTournament \in \begin{Bmatrix}
2\\ 
3\\ 
4\\ 
5\\ 
int\left( \frac{1}{3}\cdot \sqrt{CountOfFitness}\right)  \\ 
int\left( \frac{1}{2}\cdot \sqrt{CountOfFitness}\right)\\ 
int\left( \frac{2}{3}\cdot \sqrt{CountOfFitness}\right)\\ 
int\left( \sqrt{CountOfFitness}\right)
\end{Bmatrix}
\end{equation}

То есть $SizeOfTournament$ может принимать $ 8 $ значений.

Число вариантов настроек алгоритма равно \textbf{144}.

Результат исследований алгоритма можно посмотреть тут:

\href{https://github.com/Harrix/HarrixDataOfOptimizationTesting}{https://github.com/Harrix/HarrixDataOfOptimizationTesting}

В библиотеке HarrixMathLibrary данный алгоритм реализован в виде функции HML\_BinaryGeneticAlgorithmTournamentSelectionWithReturn. Библиотеку можно найти тут:

\href{https://github.com/Harrix/HarrixMathLibrary}{https://github.com/Harrix/HarrixMathLibrary}