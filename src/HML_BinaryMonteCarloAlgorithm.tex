\subsection{Метод Монте-Карло (Monte-Carlo) для решения задач на бинарных строках}\label{HarrixOptimizationAlgorithms:HML_BinaryMonteCarloAlgorithm}

\textbf{Тип алгоритма}: сравниваемый алгоритм оптимизации.

\textbf{Идентификатор:} HML\_BinaryMonteCarloAlgorithm.

\textbf{Название:} Метод Монте-Карло (Monte-Carlo) для решения задач на бинарных строках.

Наиболее простой алгоритм. Случайно генерируется по равномерному закону распределения $ CountOfFitness $ решений из допустимой области. Из них выбирается лучшее.

При сравнении с работой, например, стандартного генетического алгоритма $ CountOfFitness $ (число вычислений целевой функции) выбирается таким же, что и у стандартного генетического алгоритма.

Результат исследований алгоритма можно посмотреть тут:

\href{https://github.com/Harrix/HarrixDataOfOptimizationTesting}{https://github.com/Harrix/HarrixDataOfOptimizationTesting}

В библиотеке HarrixMathLibrary данный алгоритм реализован в вде функции HML\_BinaryMonteCarloAlgorithm. Библиотеку можно найти тут:

\href{https://github.com/Harrix/HarrixMathLibrary}{https://github.com/Harrix/HarrixMathLibrary}